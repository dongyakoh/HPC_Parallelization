\documentclass[xcolor=x11names,compress]{beamer}
%\documentclass[xcolor=x11names,handout]{beamer}
%\documentclass{beamer}

 
 %% Beamer Layout %%%%%%%%%%%%%%%%%%%%%%%%%%%%%%%%%%
\useoutertheme[subsection=false]{miniframes}
\useinnertheme{default}
\usefonttheme{serif}
\usepackage{palatino}
\usepackage{comment}

\setbeamerfont{title like}{shape=\scshape}
\setbeamerfont{frametitle}{shape=\scshape, size=\large, series=\bfseries}

%\setbeamercolor*{lower separation line head}{bg=DeepSkyBlue4} 
\setbeamercolor*{normal text}{fg=black,bg=white} 
\setbeamercolor*{alerted text}{fg=red} 
\setbeamercolor*{example text}{fg=black} 
\setbeamercolor*{structure}{fg=black} 
 
\setbeamercolor*{palette tertiary}{fg=black,bg=black!10} 
\setbeamercolor*{palette quaternary}{fg=black,bg=black!10} 

\renewcommand{\(}{\begin{columns}}
\renewcommand{\)}{\end{columns}}
\newcommand{\<}[1]{\begin{column}{#1}}
\renewcommand{\>}{\end{column}}
%%%%%%%%%%%%%%%%%%%%%%%%%%%%%%%%%%%%%%%%%%%%%%%%%%

\usepackage{graphicx}    % needed for including graphics e.g. EPS, PS
\usepackage{amsmath}
\usepackage{amsthm}
\usepackage{tabu}
\usepackage[font=small]{caption}
\usepackage[position=top]{subfig}
\usepackage{tabularx}
\usepackage{hyperref}
\usepackage{multirow}
\usepackage{tikz}
\usetikzlibrary{arrows,shapes}
\usetikzlibrary{decorations.pathreplacing}
\usepackage{etex}
\usepackage{comment}

\hypersetup{colorlinks,linkcolor=}

\setbeamercovered{transparent}
\setbeamertemplate{itemize item}[triangle]
\setbeamertemplate{itemize subitem}[circle]

\newcommand{\se}{\tiny}
\newcommand{\backupbegin}{
   \newcounter{framenumberappendix}
   \setcounter{framenumberappendix}{\value{framenumber}}
}
\newcommand{\backupend}{
   \addtocounter{framenumberappendix}{-\value{framenumber}}
   \addtocounter{framenumber}{\value{framenumberappendix}} 
}
\newcommand\setrow[1]{\gdef\rowmac{#1}#1\ignorespaces}
\newcommand\clearrow{\global\let\rowmac\relax}
\clearrow


\begin{document}


\title{Granularity versus Parallelization Overhead\\ in High-Performance Computing Systems}
\author{Dongya Koh}
\institute{University of Arkansas}
\date{October 9-10, 2018}

%%%%%%%%%%%%%%%%%%%%%%%%%%%%%%%%%%%%%%
\begin{frame}
  \titlepage
\end{frame}



%%%%%%%%%%%%%%%%%%%%%%%%%%%%%%%%%%%%%%
\section{Introduction}
\begin{frame}
\frametitle{Motivation}
\begin{itemize}
\item Solving a dynamic model with various kinds of heterogeneity needs to rely on numerical methods.
\item Over the decades, a number of alternative solution methods were developed to improve the computation time and accuracy.
\item The advent of multicore processors has accelerated the use of parallel computation.
\item To further speed up, high-performance computing (HPC) system is now available.
\end{itemize}
\bigskip
\textcolor{orange}{\textit{A Basic Question:}}
\begin{itemize}
\item To speed up the computational time, how can we take advantage of HPC?
\end{itemize}
\end{frame}


%%%%%%%%%%%%%%%%%%%%%%%%%%%%%%%%%%%%%%
\begin{frame}
\frametitle{How can we take advantage of HPC?}
The right choice of
\bigskip
\begin{enumerate}
\item Programming languages
\begin{itemize}
\item[] \textcolor{gray}{C/C++, Fortran, Matlab, Python, Julia, R, etc.}
\smallskip
\end{itemize}
\item Solution methods
\begin{itemize}
\item[] \textcolor{gray}{VFI, PFI, Projection, Perturbation, EGM, etc.}
\smallskip
\end{itemize}
\item Scalability (how effectively parallelizable)
\item Granularity (the amount of work in the parallel task)
\item Number of processing elements/cores
\end{enumerate}
\end{frame}



%%%%%%%%%%%%%%%%%%%%%%%%%%%%%%%%%%%%%%
\begin{frame}
\frametitle{Programming Languages }
\begin{center}
\small
\begin{tabular}{p{2cm} | p{5cm} | p{3cm}}
\hline
Language & Type & Rel. Exec. Time\\[0.1em]
\hline \hline
&&\\[-0.8em]
C/C++ 	& low-level, fastest with gcc & 1.00 \\[0.2em]
Fortran & low-level  & 1.05 \\[0.2em] 
\hline
&&\\[-0.8em]
Python 	& high-level, open-source, growing in popularity & 44$\times$ $\sim$ 270$\times$ \\[0.2em]
Julia 	& high-level, new open-source & 2.64$\times$ $\sim$ 2.70$\times$ \\[0.2em]
R & high-level, open-source & 281$\times$ $\sim$ 475$\times$ \\[0.2em]
Matlab & high-level, not free, license issue & 9$\times$ $\sim$ 11$\times$ \\[0.2em]
Mathematica & high-level & 809$\times$ \\[0.2em]
\hline
\end{tabular}
\end{center}
\vspace{-0.8em}
\hspace{1em}\scriptsize{Source: Arouba \& Fernandez-Villaverde (2014)}
\end{frame}


%%%%%%%%%%%%%%%%%%%%%%%%%%%%%%%%%%%%%%
\begin{frame}
\frametitle{Solution Methods}
\begin{center}
\footnotesize
\begin{tabular}{p{5cm} | p{5cm}}
\hline
\\[-1.0em]
Low Dimension & High Dimension\\
\\[-1.0em]
\hline
\\[-0.7em]
Value function iteration (VFI) & Smolyak sparse grid method\\
& \hspace{0.1em} \textcolor{gray}{{\footnotesize (Krueger \& Kubler, 2004, etc.)}} \\
Policy function iteration (PFI) & Adaptive grid method \\
& \hspace{0.1em} \textcolor{gray}{{\footnotesize (Brumm \& Scheidegger, 2017)}} \\
Projection method & Stochastic simulation algorithm \\
 \hspace{0.1em} \textcolor{gray}{{\footnotesize (Judd, 1992, etc.)}} & \hspace{0.1em} \textcolor{gray}{{\footnotesize (Den Haan \& Marcet, 1990, etc.)}} \\
Endogenous grid method (EGM) & $\varepsilon$-distinguishable set method \\
\hspace{0.1em} \textcolor{gray}{{\footnotesize (Carroll, 2005, etc.)}} & \hspace{0.1em} \textcolor{gray}{{\footnotesize (Judd, Maliar, Maliar, 2015)}} \\
Envelope condition method & Cluster grid method \\
\hspace{0.1em} \textcolor{gray}{{\footnotesize (Maliar \& Maliar, 2013)}} & \hspace{0.1em} \textcolor{gray}{{\footnotesize (Judd, Maliar, Maliar, 2015)}} \\
Precomputation method & Perturbation method \\
\hspace{0.1em} \textcolor{gray}{{\footnotesize (Judd, Maliar, Maliar, \& Tsener, 2017)}} & \hspace{0.1em} \textcolor{gray}{{\footnotesize (Judd \& Guu, 1993, etc.)}}\\
[-0.7em]\\\hline
\end{tabular}
\end{center}

\end{frame}





%%%%%%%%%%%%%%%%%%%%%%%%%%%%%%%%%%%%%%
\begin{frame}
\frametitle{Optimal Number of (Physical or Virtual) Cores}
\begin{itemize}
\item The total runtime of a program with $n$ cores:
\begin{align*}
T(n) = \frac{T_p}{n} + T_s + P(n)
\end{align*}
\item The optimal number of cores to minimize the runtime:
\begin{align*}
P'(n^*) = \frac{T_p}{n^{*2}} 
\end{align*}
\item If the parallelization overhead is approximated as $P(n)=pn^\alpha \quad (p,\alpha>0)$:
\begin{align*}
n^* = \left(\frac{T_p}{\alpha p} \right)^{1+\alpha}
\end{align*}
\end{itemize}
\end{frame}


%%%%%%%%%%%%%%%%%%%%%%%%%%%%%%%%%%%%%%
\begin{frame}
\frametitle{Parallelization Overhead}
\begin{itemize}
\item Sources of parallelization overhead:
\begin{enumerate}
\item Communication overhead in the form of synchronization and data communications
\item Idling due to load imbalances
\end{enumerate}
\item Overheads vary by the implemented parallel algorithms and problems to be solved 
\item Parallelization overhead with $n$ cores can be calculated as
\begin{align*}
P(n) = (T(n)-T_s) - \frac{T_p}{n},
\end{align*}
provided that $P(1)=0$
\end{itemize}
\end{frame}



%%%%%%%%%%%%%%%%%%%%%%%%%%%%%%%%%%%%%%
\begin{frame}
\frametitle{Overview}
\textcolor{orange}{\textit{What We Do:}}
\begin{itemize}
\item Solve a variety of life-cycle models with efficient/inefficient solution methods.
\item Compute the parallelization overhead.
\item Compare the runtime of each parallel program by granularity and the number of cores.
\end{itemize}
\bigskip
\textcolor{orange}{\textit{What We Find:}}
\begin{itemize}
\item Parallelization overhead sets an upper bound to the HPC speed-ups.
\item A design of HPC-efficient algorithm to reduce the parallelization overhead is essential for speed-ups in HPC.
\end{itemize}
\end{frame}




%%%%%%%%%%%%%%%%%%%%%%%%%%%%%%%%%%%%%%
\begin{frame}
\frametitle{HPC Environment}
\textit{HPC System:}
\begin{itemize}
\item HPC consists of three sub clusters
\item Interconnected with a 324-port QDR 40 Gbps nonblocking QLogic Infiniband switch and supplementary switches
\item 88TB long-term storage and 35 TB of scratch storage
\item OS: Centos 6.5
\end{itemize}
\smallskip
\textit{Specifications of General Use Queue:}
\begin{itemize}
\item CPU: 4x AMD Opteron 16-core 2.3 GHz 6276
\item Memory/node: 512GB
\item Max PBS spec: nodes=2:ppn=64
\item Max PBS time: 72:00:00
\item Software: Python/3.6.0-Anaconda
\end{itemize}

\end{frame}


%%%%%%%%%%%%%%%%%%%%%%%%%%%%%%%%%%%%%%
\section{Model 1}
\begin{frame}
\frametitle{Model 1: Consumption/Saving}
\begin{itemize}
\item The baseline model assumes that an individual lives until age $T$.
\item The individual's problem is to choose an amount of saving and consumption
\begin{align*}
V_t(a_{t},e_t) = \max_{c_t,a_{t+1}} u(c_{t}) + \beta \mathbb{E}_t V_{t+1}(a_{t+1},e_{t+1})
\end{align*}
\begin{align*}
\text{s.t.} \quad c_t + a_{t+1} 	&= w e_t + (1+r)a_t \\
a_{t+1} 		&\geq \underline{a}\\
e_{t+1} 		&\sim P(e_{t+1}|e_t).
\end{align*}
\item The utility function takes isoelastic preference, $u(c) = \frac{c^{1-\sigma}}{1-\sigma}$
\end{itemize}
\end{frame}



%%%%%%%%%%%%%%%%%%%%%%%%%%%%%%%%%%%%%%
\begin{frame}
\frametitle{Model 1: VFI Algorithm}
\vspace{-0.7em}
\begin{center}
\footnotesize
\begin{tabular}{p{1.0cm} p{9cm}}
\hline
Step 1. & Initialization\\
\hline
\\[-0.7em]
\hfill a.&  Set model parameters.\\
\hfill b.&  Define grid points for $(a_{t},e_{t}) \in \mathcal{A} \otimes \mathcal{E}$.\\
\hfill c.&  Construct a transition matrix $P(e_{t+1}|e_t)$.\\
\\[-0.8em]\hline
Step 2. & Computing a household problem at $t=T$\\
\hline
\\[-0.7em]
\hfill a.&  For any given $(a_T, e_T)$, $a_{T+1}=0$. Therefore, we obtain\\
[-0.9em]
&\parbox{4cm}{
\begin{align}
&c_T = w e_T + (1+r) a_T \notag\\
&V_{T}(a_T,e_T) = u(c_T)\notag
\end{align}}\\
\\[-1.7em]\hline
Step 3. & Computing a household problem at $t<T$\\
\hline
\\[-0.7em]
\hfill \textcolor<2->{red}{a.}&  \textcolor<2->{red}{For each point $(a_{t},e_{t}) $ with $V_{t+1}$, compute for all $a_{t+1}$,} \\
[-0.9em]
&\textcolor<2->{red}{
\parbox{4cm}{
\begin{align}
W_t(a_{t},e_t) = u(w e_t + (1+r)a_t - a_{t+1}) + \beta \mathbb{E}_t V_{t+1}(a_{t+1},e_{t+1}) \notag
\end{align}}} \\
[-0.5em]
\hfill \textcolor<2->{red}{b.}&  \textcolor<2->{red}{Then, choose}\\
[-0.9em]
&\textcolor<2->{red}{
\parbox{4cm}{
\begin{align}
\max_{a_{t+1}} W_t(a_t,e_t) = V_t(a_t,e_t). \notag
\end{align}}}\\
\\[-1.7em]\hline
\end{tabular}
\end{center}
\end{frame}


%%%%%%%%%%%%%%%%%%%%%%%%%%%%%%%%%%%%%%
\setcounter{subfigure}{0}
\begin{frame}
\frametitle{Model 1: VFI Running Time by Cores}
\begin{figure}
\begin{center}
\includegraphics[width=0.9\textwidth]{Graphs/m1_vfi_by_cores.png}
\end{center}
\end{figure}
\end{frame}



%%%%%%%%%%%%%%%%%%%%%%%%%%%%%%%%%%%%%%
\setcounter{subfigure}{0}
\begin{frame}
\frametitle{Model 1: VFI Running Time and Accuracy by Grids}
\begin{figure}
\begin{center}
\includegraphics[width=0.9\textwidth]{Graphs/m1_vfi_by_cores_grids.png}
\end{center}
\end{figure}
\end{frame}



%%%%%%%%%%%%%%%%%%%%%%%%%%%%%%%%%%%%%%
\begin{frame}
\frametitle{Model 1: EGM Algorithm}
\vspace{-0.7em}
\begin{center}
\footnotesize
\begin{tabular}{p{1cm} p{9cm}}
\hline
Step 1. & Initialization\\
\hline
\\[-0.8em]
& $\cdots$\\
\\[-0.8em]\hline
Step 2. &(EXGM) Computing a household problem at $t=T$\\
\hline
\\[-0.8em]
\hfill a.& For any given $(a_T, e_T)$, $a_{T+1}=0$. Therefore, we obtain\\
[-0.9em]
&\parbox{4cm}{
\begin{align}
&c_T = w e_T + (1+r) a_T \notag\\
&V^a_{T} = u'(c_T)(1+r)\notag
\end{align}}\\
\\[-1.7em]\hline
Step 3. &(ENGM/EXGM) Computing a household problem at $t<T$\\
\hline
\\[-0.8em]
\hfill a.& (FOC) $\beta \mathbb{E} V^a_{t+1} = u'(c_t)$: compute $\hat{c}_t$ for each point in $(a_{t+1},e_{t})$.\\
\hfill b.&  (BC) $c_{t} 	= w e_t + (1+r)a_t - a_{t+1}$: compute $\hat{a}_t$ for each $(\hat{c}_t,a_{t+1},e_t)$.\\
\hfill c.&  Given the obtained policy functions $g^c_t(\hat{a}_t,e_t) = \hat{c}_t$, $g^a_{t+1}(\hat{a}_t,e_t) = a_{t+1}$, \textcolor<2->{red}{interpolate new policy functions for each grid point $(a_t,e_t)$.}\\
\hfill d.&  (EC) $V^a_t = u'(c_t) (1+r)$, we compute $V^a_{t}$ . \\
\\[-0.8em]\hline
\end{tabular}
\end{center}
\end{frame}


%%%%%%%%%%%%%%%%%%%%%%%%%%%%%%%%%%%%%%
\setcounter{subfigure}{0}
\begin{frame}
\frametitle{Model 1: EGM Running Time by Cores}
\begin{figure}
\begin{center}
\includegraphics[width=.9\textwidth]{Graphs/m1_egm_by_cores.png}
\end{center}
\end{figure}
\end{frame}




%%%%%%%%%%%%%%%%%%%%%%%%%%%%%%%%%%%%%%
\setcounter{subfigure}{0}
\begin{frame}
\frametitle{Model 1: EGM Running Time and Accuracy by Grids}
\begin{figure}
\begin{center}
\includegraphics[width=0.9\textwidth]{Graphs/m1_egm_by_cores_grids.png}
\end{center}
\end{figure}
\end{frame}



%%%%%%%%%%%%%%%%%%%%%%%%%%%%%%%%%%%%%%
\section{Model 2}
\begin{frame}
\frametitle{Model 2: Elastic Labor Supply}
\begin{itemize}
\item The individual's problem is to choose an amount of saving, consumption, and labor supply.
\begin{align*}
V_t(a_{t},e_t) = \max_{c_t,n_t,a_{t+1}} u(c_{t},n_t) + \beta \mathbb{E}_t V_{t+1}(a_{t+1},e_{t+1})
\end{align*}
\begin{align*}
\text{s.t.} \quad c_t + a_{t+1} 	&= w e_t n_t + (1+r)a_t \\
a_{t+1} 		&\geq \underline{a}\\
e_{t+1} 		&\sim P(e_{t+1}|e_t).
\end{align*}
\item the utility function takes isoelastic preference, $u(c,n) = \frac{c^{1-\sigma}}{1-\sigma} - \chi \frac{n^{1+\eta}}{1+\eta}$.
\end{itemize}
\end{frame}




%%%%%%%%%%%%%%%%%%%%%%%%%%%%%%%%%%%%%%
\begin{frame}
\frametitle{Model 2: EGM Algorithm}
\vspace{-0.7em}
\begin{center}
\footnotesize
\begin{tabular}{p{1cm} p{9cm}}
\hline
Step 1. & Initialization\\
\hline
\\[-0.8em]
& $\cdots$\\
\\[-0.8em]\hline
Step 2. &(EXGM) Computing a household problem at $t=T$\\
\hline
\\[-0.8em]
\hfill a.& For each point $(a_T, e_T)$, $a_{T+1}=0$.\\
\hfill \textcolor<2->{red}{b.}& \textcolor<2->{red}{(FOC1) $\chi n_T^{\eta} = w e_T (w e_T n_T + (1+r)a_T)^{-\sigma}$: solve a nonlinear problem for $\hat{n}_{T}$}\\
\hfill c.& (BC) $c_T = w e_T n_{T} + (1+r) a_T$: compute $\hat{c}_T$.\\
\hfill d.& (EC) $V^a_T(a_{T},e_T) = u_c(c_T,n_T)(1+r)$: compute $V^a_T$.\\
\\[-0.8em]\hline
Step 3. &(ENGM/EXGM) Computing a household problem at $t<T$\\
\hline
\\[-0.8em]
\hfill a.& (FOC2) $c_t^{-\sigma} = \beta \mathbb{E}V^a_{t+1}$: compute $\hat{c}_t$ for each point in $(a_{t+1},e_{t})$.\\
\hfill b.& From (FOC1), compute $\hat{n}_t$.\\
\hfill c.& (BC) $a_{t+1} = w e_t n_t + (1+r)a_t - c_t$: compute $\hat{a}_t$.\\
\hfill d.& Given the obtained policy functions $g^c_t(\hat{a}_t,e_t) = \hat{c}_t$, $g^n_t(\hat{a}_t,e_t) = \hat{n}_t$, $g^a_{t+1}(\hat{a}_t,e_t) = a_{t+1}$, we interpolate new policy functions, $g^c_{t}(a_t,e_t) = \tilde{c}_{t}$, $g^n_{t}(a_t,e_t) = \tilde{n}_{t}$, and $g^a_{t+1}(a_t,e_t) = \tilde{a}_{t+1}$ for each grid point $(a_t,e_t)\in \mathcal{A} \otimes \mathcal{E}$.\\
\hfill e.& If $g^a_{t+1}(a_t,e_t) = \tilde{a}_{t+1} < \underline{a}$, then do Step 2 only for the specific $(a_t,e_t)\in \mathcal{A} \otimes \mathcal{E}$. Otherwise, we compute $V^a_{t}$ from (EC).\\
\\[-0.8em]\hline
\end{tabular}
\end{center}
\end{frame}



%%%%%%%%%%%%%%%%%%%%%%%%%%%%%%%%%%%%%%
\setcounter{subfigure}{0}
\begin{frame}
\frametitle{Model 2: EGM Running Time by Cores}
\begin{figure}
\begin{center}
\includegraphics[width=.9\textwidth]{Graphs/m2_egm_by_cores.png}
\end{center}
\end{figure}
\end{frame}




%%%%%%%%%%%%%%%%%%%%%%%%%%%%%%%%%%%%%%
\setcounter{subfigure}{0}
\begin{frame}
\frametitle{Model 2: EGM Running Time and Accuracy by Grids}
\begin{figure}
\begin{center}
\includegraphics[width=0.9\textwidth]{Graphs/m2_egm_by_cores_grids.png}
\end{center}
\end{figure}
\end{frame}





%%%%%%%%%%%%%%%%%%%%%%%%%%%%%%%%%%%%%%
\section{Model 3}
\begin{frame}
\frametitle{Model 3: Human Capital Investment}
\begin{itemize}
\item The individual's problem is to choose an amount of saving, consumption, and human capital investment
\begin{align*}
V_t(a_{t},h_{t},e_t) = \max_{c_t,a_{t+1},s_{t},h_{t+1}} u(c_{t}) + \beta \mathbb{E} V_{t+1}(a_{t+1},h_{t+1},e_{t+1})
\end{align*}
\begin{align*}
\text{s.t.} \quad c_t + a_{t+1} 	&= w e_t h_t (1-s_t) + (1+r)a_t \\
h_{t+1} 		&= (1-\delta)h_t + A_h h_t^\alpha s_t^\gamma\\
s_t 			& \in [0,1]\\
a_{t+1} 		&\geq \underline{a}\\
e_{t+1} 		&\sim P(e_{t+1}|e_t).
\end{align*}
\item The utility function takes isoelastic preference, $u(c) = \frac{c^{1-\sigma}}{1-\sigma}$
\end{itemize}
\end{frame}




%%%%%%%%%%%%%%%%%%%%%%%%%%%%%%%%%%%%%%
\begin{frame}
\frametitle{Model 3: EGM Algorithm}
\vspace{-0.7em}
\begin{center}
\footnotesize
\begin{tabular}{p{1cm} p{9cm}}
\hline
Step 1. & Initialization\\
\hline
\\[-0.8em]
& $\cdots$\\
\\[-0.8em]\hline
Step 2. &(EXGM) Computing a household problem at $t=T$\\
\hline
\\[-0.8em]
\hfill a.& For each point $(a_T, h_T, e_T)$, $a_{T+1}=s_T=0$.\\
\hfill b.& (BC) $c_T = w e_T h_T + (1+r) a_T$: compute $\hat{c}_T$.\\
\hfill c.& (EC1) $V^a_{T} = u'(c_T)(1+r)$: compute $V^a_T$.\\
\hfill d.& (EC2) $V^h_{T} = u'(c_T) w e_T \left(1 - \left(1-\frac{\alpha}{\gamma}\right) s_T + \frac{1-\delta}{\gamma A_h h_T^{\alpha-1} s_T^{\gamma -1}}\right)$: \\
& compute $V^h_{T}$.\\
\\[-0.5em]\hline
\end{tabular}
\end{center}
\end{frame}



%%%%%%%%%%%%%%%%%%%%%%%%%%%%%%%%%%%%%%
\begin{frame}
\frametitle{Model 3: EGM Algorithm}
\vspace{-0.7em}
\begin{center}
\footnotesize
\begin{tabular}{p{1cm} p{9cm}}
\hline
Step 3. &(ENGM/EXGM) Computing a household problem at $t<T$\\
\hline
\\[-0.5em]
\hfill \textcolor<2->{red}{a.}& \textcolor<2->{red}{(FOC1) $\mathbb{E} V^a_{t+1}  w e_t h_t = \mathbb{E} V^h_{t+1} \gamma A_h h_t^\alpha s_t^{\gamma-1}$: solve a root-finding problem for $\hat{s}_t$ for each point $(a_{t+1},h_t,e_{t})$.}\\
\hfill b.& (FOC2) $u'(c_t) = \beta \mathbb{E} V^a_{t+1}$: compute $\hat{c}_t$.\\
\hfill c.& (BC) $c_t + a_{t+1} 	= w e_t h_t (1-s_t) + (1+r)a_t$: compute $\hat{a}_t$.\\
\hfill d.& Given the obtained policy functions $g^c_t(\hat{a}_t,h_t,e_t) = \hat{c}_t$, $g^s_t(\hat{a}_t,h_t,e_t) = \hat{s}_t$, $g^a_{t+1}(\hat{a}_t,h_t,e_t) = a_{t+1}$, interpolate new policy functions, $g^c_{t}(a_t,h_t,e_t) = \tilde{c}_{t}$, $g^s_{t}(a_t,h_t,e_t) = \tilde{s}_{t}$, and $g^a_{t+1}(a_t,h_t,e_t) = \tilde{a}_{t+1}$ for each grid point $(a_t,h_t,e_t)$.\\
\hfill e.& If $g^a_{t+1}(a_t,h_t,e_t) = \tilde{a}_{t+1} < \underline{a}$, then do Step 3(a) only for the specific $(a_t,h_t,e_t)$ and $a_{t+1}=\underline{a}$. Otherwise, we compute $V^a_{t}$ and $V^h_{t}$ from (EC1) and (EC2).\\
\\[-0.5em]\hline
\end{tabular}
\end{center}
\end{frame}



%%%%%%%%%%%%%%%%%%%%%%%%%%%%%%%%%%%%%%
\setcounter{subfigure}{0}
\begin{frame}
\frametitle{Model 3: EGM Running Time by Cores}
\begin{figure}
\begin{center}
\includegraphics[width=.9\textwidth]{Graphs/m3_egm_by_cores.png}
\end{center}
\end{figure}
\end{frame}



%%%%%%%%%%%%%%%%%%%%%%%%%%%%%%%%%%%%%%
\setcounter{subfigure}{0}
\begin{frame}
\frametitle{Model 3: EGM Running Time and Accuracy by Grids}
\begin{figure}
\begin{center}
\includegraphics[width=0.9\textwidth]{Graphs/m3_egm_by_cores_grids.png}
\end{center}
\end{figure}
\end{frame}





%%%%%%%%%%%%%%%%%%%%%%%%%%%%%%%%%%%%%%
\section{Conclusion}
\begin{frame}
\frametitle{Technical Issues Using Matlab \& Python in HPC}
\textit{Matlab:}
\begin{itemize}
\item The drawback of the Parallel Computing Toolbox is that it contains digital restrictions that only permit it to operate with 12 cores or fewer on a single computer.
\item Matlab Distributed Computing Server costs several thousand dollars depending on the number of nodes licensed.
\end{itemize}
\bigskip
\textit{Python:}
\begin{itemize}
%\item The parallelization overhead is surprisingly higher with two cores than with more than 2 cores.
\item A joint use of numba and multiprocessing packages has a systemic conflict.
\end{itemize}
\end{frame}



%%%%%%%%%%%%%%%%%%%%%%%%%%%%%%%%%%%%%%
\begin{frame}
\frametitle{Conclusion}
\begin{itemize}
\item \textit{Efficiently} running a parallel program in HPC environment requires a great amount of engineering effort.
\item Parallelization overhead sets an upper bound to the HPC speed-ups.
\item A design of HPC-efficient algorithm to reduce the parallelization overhead is essential for speed-ups in HPC.
\end{itemize}
\end{frame}




\end{document}






%%%%%%%%%%%%%%%%%%%%%%%%%%%%%%%%%%%%%%
%%%%%%%%%%%%%%%%%%%%%%%%%%%%%%%%%%%%%%
%%%%%%%%%%%%%%%%%%%%%%%%%%%%%%%%%%%%%%
%%%%%%%%%%%%%%%%%%%%%%%%%%%%%%%%%%%%%%
%%%%%%%%%%%%%%%%%%%%%%%%%%%%%%%%%%%%%%
%%%%%%%%%%%%%%%%%%%%%%%%%%%%%%%%%%%%%%
%%%%%%%%%%%%%%%%%%%%%%%%%%%%%%%%%%%%%%
%%%%%%%%%%%%%%%%%%%%%%%%%%%%%%%%%%%%%%
%%%%%%%%%%%%%%%%%%%%%%%%%%%%%%%%%%%%%%
%%%%%%%%%%%%%%%%%%%%%%%%%%%%%%%%%%%%%%
%%%%%%%%%%%%%%%%%%%%%%%%%%%%%%%%%%%%%%
%%%%%%%%%%%%%%%%%%%%%%%%%%%%%%%%%%%%%%
%%%%%%%%%%%%%%%%%%%%%%%%%%%%%%%%%%%%%%
%%%%%%%%%%%%%%%%%%%%%%%%%%%%%%%%%%%%%%
%%%%%%%%%%%%%%%%%%%%%%%%%%%%%%%%%%%%%%
%%%%%%%%%%%%%%%%%%%%%%%%%%%%%%%%%%%%%%
%%%%%%%%%%%%%%%%%%%%%%%%%%%%%%%%%%%%%%
%%%%%%%%%%%%%%%%%%%%%%%%%%%%%%%%%%%%%%

%%%%%%%%%%%%%%%%%%%%%%%%%%%%%%%%%%%%%%
\title{Constrained Inefficiency over the Life Cycle}
\author{Dongya (Don) Koh}
\institute{University of Arkansas}
\begin{frame}
  \titlepage
\end{frame}



%%%%%%%%%%%%%%%%%%%%%%%%%%%%%%%%%%%%%%
\section{Motivation}
\begin{frame}
\frametitle{Motivation}
\small
\begin{itemize}
\item In a standard incomplete market model with uninsurable idiosyncratic shocks, the market equilibrium is inefficient.
\item \textcolor{blue}{\textit{Constrained efficiency}} is an allocation in which the planner improves market allocations by internalizing the effect on market prices without changing market structures and household budget constraints.
\item Under- or over- accumulation of capital is determined by the factor income compositions of the consumption-poor. {\footnotesize (D\'avila et al. 2012, Park 2017)}
\end{itemize}

\smallskip
\textcolor{orange}{\textit{Questions}:}
\begin{itemize}
\item Who are more responsible for the over- or under-accumulation of capital?
\item To what extent do saving and investment in human capital over an individual's course of life depart from efficient levels?
\end{itemize}
\end{frame}


%%%%%%%%%%%%%%%%%%%%%%%%%%%%%%%%%%%%%%
\begin{frame}
\frametitle{Overview}
\small
\textcolor{orange}{\textit{What We Do}:}
\begin{itemize}
\item We quantitatively explore the extent to which the profiles of saving and human capital investment of U.S. households depart from the efficient levels.
\end{itemize}
\textcolor{orange}{\textit{What We Find}:}
\begin{itemize}
\item The median U.S. households save an average of 18.5\% less.
\item They invest 11\% more time in human capital over the life-cycle.
\item The top 10\% of income earners should save 26\% more and invest 17.3\% less time in human capital.
\item The bottom 10\% of income earners throughout their lives are almost at the efficient level.
\end{itemize}
\textcolor{orange}{\textit{If time allows...}:}
\begin{itemize}
\item Policy implications (tax/subsidy implementations)
\item Life-cycle effects of student debts
\end{itemize}
\end{frame}



%%%%%%%%%%%%%%%%%%%%%%%%%%%%%%%%%%%%%%
\begin{frame}[label=cons-poor]
\frametitle{The Fraction of the Bottom Ten Percent of Households}
\begin{figure}
\begin{center}
\includegraphics[width=.8\textwidth]{Graphs/frac_age.png}
\end{center}
\end{figure}
\hyperlink{lifecycle-prop}{\beamergotobutton{Life-Cycle Properties}}
\end{frame}



\section{Model}

%%%%%%%%%%%%%%%%%%%%%%%%%%%%%%%%%%%%%%
\begin{frame}[label=household]
\frametitle{Benchmark Model: Household's Problem}
\begin{itemize}
\item Given factor prices, the household policy functions solve the recursive household problem, for $j\leq J_R$,
{\footnotesize
\begin{align*}
V_j(a_j,h_j;\theta_0) &= \max_{c_j,n_j,s_j,a_{j+1},h_{j+1}} u(c_j,1-n_j) + \beta \mathbb{E} V_{j+1}(a_{j+1},h_{j+1};\theta_0) \\
\text{s.t.} \quad &(1+\tau_c) c_j + a_{j+1} = \tilde{y}_j - T(\tilde{y}_j)  + (1+r(1-\tau_{k}))a_j \\
& h_{j+1} = \exp(\varepsilon_{j+1})H(h_j,n_j,s_j;\theta_0) \\
&\tilde{y}_j =w h_j n_j (1-s_j) - 0.5\tau_{ss}\min(w h_j n_j (1-s_j),\bar{y})\\
&a_j \geq -\underline{a}
\end{align*}
}
\item And for $j>J_R$
{\footnotesize
\begin{align*}
W_j(a_j) &= \max_{c_j,a_{j+1}} u(c_j) + \beta  W_{j+1}(a_{j+1}) \\
\text{s.t.} \quad &(1+\tau_c) c_j + a_{j+1} = y_{ss}  + (1+r(1-\tau_{k}))a_j \\
&a_j \geq -\underline{a} 
\end{align*}
}
\item Utility function: $u(c_j,1-n_j) = \frac{c_j^{1-\psi}}{1-\psi} + \chi\frac{(1-n_j)^{1-\phi}}{1-\phi}$.
\end{itemize}
\hyperlink{model}{\beamergotobutton{Household's Problem}}
\end{frame}


%%%%%%%%%%%%%%%%%%%%%%%%%%%%%%%%%%%%%%
\begin{frame}
\frametitle{Benchmark Model: Household's Problem}
\noindent\textbf{\textit{Human Capital Accumulation:}}
\begin{itemize}
\item Technically, Ben-Porath's OJT technology gives three- to four-times higher consumption growth than income growth.
\vspace{-0.3cm}
\begin{itemize}
\item Unrealistically high consumption or capital income tax.
\item A positive initial asset that provides an extra insurance.
\item Utility from leisure.
\end{itemize}
\item We incorporate an elastic labor supply in the form of learning-by-doing.
\item Human capital accumulation:
\begin{align*}
h_{j+1} = \exp(\varepsilon_{j+1})[(1-\delta_h) h_j + \theta_0 (h_j n_j s_j)^{\gamma}]
\end{align*}
\end{itemize}
\end{frame}



\section{Competitive Equilibrium}
%%%%%%%%%%%%%%%%%%%%%%%%%%%%%%%%%%%%%%
\begin{frame}[label=comp_eq]
\frametitle{Benchmark Model: Competitive Equilibrium}
\small
\textit{A competitive equilibrium with tax distortions} is a collection of policy functions, value functions, factor prices $\{r,w\}$, aggregate input factors $\{K,L\}$, and distributions of individual states by age $\{\Psi_j(\mathbf{x}_j)\}_{j=1}^J$, given the social security payments, government spending, a set of taxes, and an initial distribution $\mathcal{N}(\mathbf{M},\mathbf{\Sigma})$, such that
\begin{enumerate}
\item Given factor prices, the household policy functions solve the recursive household problem for $j\leq J_R$,
\item Competitive factor prices equal the marginal products of the input factors in the production technology,
\item Markets clear, \quad \hyperlink{mc}{\beamergotobutton{MC}}
\item Government budgets balance, \quad \hyperlink{gb}{\beamergotobutton{Gov}}
\item and the distributions of households in each state and at each age are consistent with the household's policy functions.
\end{enumerate}
\hyperlink{calib}{\beamergotobutton{Calibration}}
\hyperlink{match}{\beamergotobutton{Data vs Model}}
\end{frame}






\section{Constrained Efficiency}
%%%%%%%%%%%%%%%%%%%%%%%%%%%%%%%%%%%%%%
\begin{frame}
\frametitle{Constrained Efficiency}
\begin{itemize}
\item The constrained optimum can be found when the planner improves on market allocations by internalizing the effect on prices without completing the market and without changing households' budget constraints.
\item The planner maximizes a utilitarian objective assigning equal weights to all households given initial states $(a_0,h_0;\theta)$:
{\footnotesize
\begin{align*}
&\max \sum_{j=0}^{J_R}  \mu_j \int_{\chi} V_j(\mathbf{x}_j)d\Psi_j + \sum_{j=J_R+1}^{J} \mu_j \int_{\chi} W_j(a_j)d\Psi_j \notag \\
\text{s.t.} \quad &K  =  \sum_{j=0}^J \mu_j \int_{\mathcal{X}} a_j d\Psi_j(\mathbf{x}_j), 
L  =  \sum_{j=0}^{J_R} \mu_j  \int_{\mathcal{X}} h_j n_j(\mathbf{x}_j) (1-s_j(\mathbf{x}_j)) d\Psi_j(\mathbf{x}_j)\\
& r = F_k(K,L), \quad w = F_l(K,L), \quad \Psi_{j+1} = G_j(\Psi_j), \quad \forall j\leq J
\end{align*}
}
\end{itemize}
\end{frame}



%%%%%%%%%%%%%%%%%%%%%%%%%%%%%%%%%%%%%%
\begin{frame}
\frametitle{Competitive Equilibrium vs. Constrained Efficiency}
\begin{table}[h!]
\footnotesize
\begin{center}
\begin{tabular}{llll}
\hline\hline
  			& \vtop{\hbox{\strut Competitive}\hbox{\strut Equilibrium}} & \vtop{\hbox{\strut Constrained}\hbox{\strut Optimum}} & \vtop{\hbox{\strut No Income}\hbox{\strut Risk}}\\ 
\hline
\\[-0.5em]
Interest rate ($r$)			& 4.22\%	& 3.44\%	& 5.68\%\\
Wage ($w$)   		   		& 1.000 	& 1.034		& 0.942\\
Aggregate output ($Y$)	    & 7.182 	& 7.249		& 5.723\\
Aggregate capital ($K$)	    & 21.126 	& 22.907	& 14.856\\
Aggregate labor	($L$)    	& 4.874 	& 4.751		& 4.121\\
Aggregate consumption ($C$)	& 4.826 	& 4.764		& 3.990\\
Capital-output ratio ($K/Y$)& 2.942 	& 3.161		& 2.596\\
Capital-labor ratio ($K/L$) & 4.335 	& 4.817		& 3.605\\
Capital wedge ($\Delta_k$)  & -		 	& 0.000145	& - \\
Labor wedge ($\Delta_h$) 	& -		 	& -0.001048	& - \\
\\[-0.5em]
\hline
\end{tabular}
\end{center}
\end{table}
\end{frame}



%%%%%%%%%%%%%%%%%%%%%%%%%%%%%%%%%%%%%%
\setcounter{subfigure}{0}
\begin{frame}
\frametitle{Life-cycle Effects: Median Household}
\begin{figure}
\begin{center}
\subfloat[\sf Saving]{\includegraphics[width=.5\textwidth]{Graphs/mean_a_eff.png}\label{fig:medassets}}
\subfloat[\sf Human Capital Investment]{\includegraphics[width=.5\textwidth]{Graphs/mean_s_eff.png}\label{fig:medtime}}
\end{center}
\end{figure}
\end{frame}




%%%%%%%%%%%%%%%%%%%%%%%%%%%%%%%%%%%%%%
\setcounter{subfigure}{0}
\begin{frame}
\frametitle{Life-cycle Effects: Median Household}
\begin{figure}
\begin{center}
\vspace{-0.5cm}
\subfloat[\sf Hourly Wage]{\includegraphics[width=.4\textwidth]{Graphs/mean_w_eff.png}\label{fig:medwage}}
\hspace{2cm}
\subfloat[\sf Consumption]{\includegraphics[width=.4\textwidth]{Graphs/mean_c_eff.png}\label{fig:medcons}}
\vspace{-0.2cm}
\subfloat[\sf Hours of Work]{\includegraphics[width=.4\textwidth]{Graphs/mean_n_eff.png}\label{fig:medwork}}
\end{center}
\end{figure}
\end{frame}



%%%%%%%%%%%%%%%%%%%%%%%%%%%%%%%%%%%%%%
\setcounter{subfigure}{0}
\begin{frame}
\frametitle{Distributional Effects: Top \& Bottom 10\% Income Earners}
\begin{figure}
\begin{center}
\subfloat[\sf Saving]{\includegraphics[width=.5\textwidth]{Graphs/mean_a_init.png}\label{fig:distassets}}
\subfloat[\sf Human Capital Investment]{\includegraphics[width=.5\textwidth]{Graphs/mean_s_init.png}\label{fig:disttime}}
\end{center}
\end{figure}
\end{frame}




%%%%%%%%%%%%%%%%%%%%%%%%%%%%%%%%%%%%%%
\setcounter{subfigure}{0}
\begin{frame}
\frametitle{Distributional Effects: Top \& Bottom 10\% Income Earners}
\begin{figure}
\begin{center}
\vspace{-0.5cm}
\subfloat[\sf Hourly Wage]{\includegraphics[width=.4\textwidth]{Graphs/mean_w_init.png}\label{fig:distwage}}
\hspace{2cm}
\subfloat[\sf Consumption]{\includegraphics[width=.4\textwidth]{Graphs/mean_c_init.png}\label{fig:distcons}}
\vspace{-0.2cm}
\subfloat[\sf Hours of Work]{\includegraphics[width=.4\textwidth]{Graphs/mean_n_init.png}\label{fig:distwork}}
\end{center}
\end{figure}
\end{frame}



%%%%%%%%%%%%%%%%%%%%%%%%%%%%%%%%%%%%%%
\setcounter{subfigure}{0}
\begin{frame}
\frametitle{Tax Implementation of Constrained Efficiency}
\begin{itemize}
\item The characterized constrained efficient profiles can also be implemented by a modified current tax system.
\item The constrained inefficiency should be incorporated into the capital income tax and progressive labor income tax with a lump-sum transfer.
\begin{align*}
&\widehat{\tau}_{k,j}(\mathbf{x}_j)  = \tau_{k} - \frac{\Delta_{k}}{r \lambda^c_j}\\
&\widehat{T}'_j(\tilde{y}_j, \mathbf{x}_j) = T'(\tilde{y}_j) - \frac{\Delta_{h}}{w \lambda^c_j (1-0.5 \tau_{ss} \mathbb{I}_{y_j\leq \bar{y}})}.
\end{align*}
\end{itemize}
\end{frame}



%%%%%%%%%%%%%%%%%%%%%%%%%%%%%%%%%%%%%%
\setcounter{subfigure}{0}
\begin{frame}
\frametitle{Tax Implementation of Constrained Efficiency}
\begin{figure}
\begin{center}
\subfloat[\sf Capital Income Tax]{\includegraphics[width=.5\textwidth]{Graphs/mean_tk_eff.png}\label{fig:taxcapital}}
\subfloat[\sf Marginal Labor Income Tax]{\includegraphics[width=.5\textwidth]{Graphs/mean_tpy_eff.png}\label{fig:taxlabor}}
\end{center}
\end{figure}
\end{frame}



%%%%%%%%%%%%%%%%%%%%%%%%%%%%%%%%%%%%%%
\begin{frame}
\frametitle{Conclusion}
\begin{itemize}
\item In the life-cycle setting, the consumption-poor are in general younger cohorts.
\item Younger cohorts become better off by increasing aggregate capital-labor ratio.
\end{itemize}
\smallskip
\textcolor{orange}{\textit{For the constrained optimum,}}
\begin{itemize}
\item Through life-cycle,
\begin{itemize}
\item Income should be transferred from old age to youth.
\item Wealth should be transferred from youth to old age.
\end{itemize}
\item Cross-sectionally,
\begin{itemize}
\item The constrained inefficiency is largely corrected by the top income earners. 
\item The lower income earners reap benefits through the change in market prices.
\end{itemize}
\end{itemize}
\end{frame}




%%%%%%%%%%%%%%%%%%%%%%%%%%%%%%%%%%%%%%%%%%%%%%%%%%%%%%%%
%%%%%%%%%%%%%%%%%%%%%%%%%%%%%%%%%%%%%%%%%%%%%%%%%%%%%%%%
%%%%%%%%%%%%%%%%%%%%%%%%%%%%%%%%%%%%%%%%%%%%%%%%%%%%%%%%
%%%%%%%%%%%%%%%%%%%%%%%%%%%%%%%%%%%%%%%%%%%%%%%%%%%%%%%%
\appendix
\backupbegin


%%%%%%%%%%%%%%%%%%%%%%%%%%%%%%%%%%%%%%
\begin{frame}
\frametitle{\sf }
\sf 
\centering
%\vspace{0.5cm}
{\huge \hfill Appendix \hfill}
\end{frame}




%%%%%%%%%%%%%%%%%%%%%%%%%%%%%%%%%%%%%%
\begin{frame}[label=lifecycle-prop]
\frametitle{Life-cycle Properties of Constrained Inefficiency}
\small
\begin{itemize}
\item \textcolor{blue}{\textit{Constrained efficiency}} is an allocation in which the planner improves market allocations by internalizing the effect on market prices without changing market structures and household budget constraints.
\smallskip
\item Extending DHKR's two-period framework to $J>2$ periods, the planner solves the following problem:
\end{itemize}
\begin{align*}
\max_{a_{j+1}} \mathbb{E}\sum_{j=0}^J \int_{\chi} \beta^j u(f_{l}(K,L) \varepsilon_j + f_{k}(K,L) a_j - a_{j+1}) d\psi_j
\end{align*}
\begin{align*}
\text{subject to}\quad &K = \sum_{j=0}^J \int_{\chi} a_{j} \; d\psi_j, \quad L = \sum_{j=0}^J \int_{\chi} \varepsilon_j \; d\psi_j,\\
&\psi_{j+1} = \int_{\chi} Q_j((a_j,\varepsilon_j),B) d\psi_j.
\end{align*}
\end{frame}


%%%%%%%%%%%%%%%%%%%%%%%%%%%%%%%%%%%%%%
\begin{frame}
\frametitle{Life-cycle Properties of Constrained Inefficiency}
\small
\begin{itemize}
\item A typical Euler equation plus an additional marginal benefit/cost from the change in market prices:
\end{itemize}
\begin{align*}
u'(c_j) = \beta  f_k(K',L) \mathbb{E} u'(c_{j+1}) + \Delta_k,
\end{align*}
where 
{\scriptsize
\begin{align*}
\Delta_k &=  \sum_{j=0}^J \beta f_{kk}(K',L) K' \int_{\chi} \mathbb{E} u'(c_{j+1}) \left[\frac{a_{j+1}}{K'}  - \frac{\varepsilon_{j+1}}{L} \right] d \psi_{j+1}
\end{align*}
}
\begin{itemize}
\item The constrained inefficiency is a marginal-utility weighted average of an individual's effect on increasing capital through market prices.
\end{itemize}
\hyperlink{cons-poor}{\beamergotobutton{Back}}
\end{frame}



%%%%%%%%%%%%%%%%%%%%%%%%%%%%%%%%%%%%%%
\begin{frame}[label=model]
\frametitle{Benchmark Model: Household's Problem}
\noindent\textbf{\textit{Per-period decisions:}}
\begin{itemize}
\item Households are \textit{ex ante} heterogeneous in their initial human capital ($h_0$), initial wealth ($a_0$), and learning ability ($\theta_0$).
\item Households are employed throughout their lives until retirement.
\item a household chooses how much to consume ($c_{j}$) and save ($a_{j+1}$). Also, at each working age $j\leq J_{R}$, one unit of time endowment is allocated into labor and leisure ($n_j\in[0,1]$), and a fraction of working hours are allocated to human capital investment ($s_j\in[0,1]$).
\item There is a borrowing limit $\underline{a}\geq 0$.
\end{itemize}
\end{frame}


%%%%%%%%%%%%%%%%%%%%%%%%%%%%%%%%%%%%%%
\begin{frame}
\frametitle{Benchmark Model: Household's Problem}
\noindent\textbf{\textit{Labor income risks:}}
\begin{itemize}
\item Households encounter idiosyncratic labor income shocks ($\varepsilon_j$) through human capital at each working age.
\item $\varepsilon_j \sim N(0,\sigma^2_{\varepsilon})$
\item The shock is uninsurable due to the absence of full insurance contracts.
\end{itemize}
\end{frame}




%%%%%%%%%%%%%%%%%%%%%%%%%%%%%%%%%%%%%%
\begin{frame}
\frametitle{Benchmark Model: Household's Problem}
\noindent\textbf{\textit{Taxes and transfers:}}
\begin{itemize}
\item The government imposes ($\tau_{c}$, $\tau_{k}$, $\tau_{ss}$, $T(y)$).
\begin{align*}
T(y) = \tau_0 (y - (y^{-\tau_1} + \tau_2)^{-1/\tau_1})
\end{align*}
\item Households receive social security payments ($y_{ss}$).
\end{itemize}
\bigskip
\noindent\textbf{\textit{Household budget constraint}}
\begin{itemize}
\item the household budget constraint for age cohort $j$ at each point in time:
\begin{align*}
(1+\tau_c) c_j + a_{j+1} &= \tilde{y}_j - T(\tilde{y}_j)  + (1+r(1-\tau_k)) a_j, & \text{ for }j\leq J_R\\
(1+\tau_c) c_j + a_{j+1} &= y_{ss}  + (1+r(1-\tau_k))a_j, & \text{ for }j>J_R.
\end{align*}
\item The household's labor income:
\begin{align*}
\tilde{y}_j = \left\{ 
\begin{array}{l l}
w  h_j n_j (1-s_j) - 0.5\tau_{ss}\min(w h_j n_j (1-s_j),\bar{y}) & \text{ if }j\leq J_R\\
0 & \text{ if }j> J_R.
\end{array}
\right.
\end{align*}
\end{itemize}
\hyperlink{household}{\beamergotobutton{Back}}
\end{frame}


%%%%%%%%%%%%%%%%%%%%%%%%%%%%%%%%%%%%%%
\begin{frame}[label=mc]
\frametitle{Benchmark Model: Aggregation}
\noindent\textbf{\textit{Production technology}}
\begin{align*}
Y = AK^\alpha L^{1-\alpha}
\end{align*}
\bigskip
\noindent\textbf{\textit{Initial Distribution}}\\
An individual's initial assets, human capital, and learning ability are trivariate log-normally distributed:
\begin{align*}
\mathbf{M}=\left(\begin{array}{c} m_a\\m_h\\m_{\theta} \end{array}\right), \quad 
\mathbf{\Sigma}=\left(\begin{array}{ccc}
\sigma^2_a 									& \lambda_{ah}\sigma_{h}\sigma_{a} 				&\lambda_{a\theta}\sigma_{\theta}\sigma_{a}\\ 
\lambda_{ah}\sigma_{a}\sigma_{h} 			& \sigma^2_h 									&\lambda_{h\theta}\sigma_{\theta}\sigma_{h}\\ 
\lambda_{a\theta}\sigma_{a}\sigma_{\theta} 	& \lambda_{h\theta}\sigma_{h}\sigma_{\theta} 	& \sigma^2_{\theta}\\ 
\end{array}\right).
\end{align*}
\end{frame}



%%%%%%%%%%%%%%%%%%%%%%%%%%%%%%%%%%%%%%
\begin{frame}
\frametitle{Benchmark Model: Aggregation}
\begin{itemize}
\item At each point in time, a household's state at age $j$: $\mathbf{x}_j = (a_j,h_j;\theta_0)$.
\item Define a state space as $\mathcal{X}=\mathcal{A}\times\mathcal{H}\times\Theta_0$.
\item Define a joint probability measure $\Psi_j$ over the probability space $(\mathcal{X},B(\mathcal{X}),\Psi_j)$
\item In equilibrium, the goods, labor, and capital markets clear at the competitive prices at each point in time:
{\footnotesize
\begin{align*}
&K  =  \sum_{j=0}^J \mu_j \int_{\mathcal{X}} a_j d\Psi_j(\mathbf{x}_j)\\
&L  =  \sum_{j=0}^{J_R} \mu_j  \int_{\mathcal{X}} h_j n_j(\mathbf{x}_j) (1-s_j(\mathbf{x}_j)) d\Psi_j(\mathbf{x}_j)\\
& \sum_{j=0}^{J} \mu_j  \int_{\mathcal{X}} c_j(\mathbf{x}_j) d\Psi_j(\mathbf{x}_j) + (1+n)K' + \mathcal{G} = Y + (1-\delta)K 
\end{align*}
}
\end{itemize}
\hyperlink{comp_eq}{\beamergotobutton{Competitive Equilibrium}}
\end{frame}


%%%%%%%%%%%%%%%%%%%%%%%%%%%%%%%%%%%%%%
\begin{frame}[label=gb]
\frametitle{Benchmark Model: Government Budget Balance}
\begin{itemize}
\item The government budget balances:
{\footnotesize
\begin{align*}
\mathcal{G} =  \sum_{j=0}^{J} \int_{\mathcal{X}} (\tau_c c_j(\mathbf{x}_j) + r \tau_{k} a_j + T(\tilde{y}_j(\mathbf{x}_j))) d\Psi_{j}(\mathbf{x}_j)
\end{align*}
}
\item The social security policies satisfy:
{\footnotesize
\begin{align*}
\sum_{j=0}^{J_R} \int_{\mathcal{X}} \tau_{ss}\min\{w h_j n_j (1-s_j(\mathbf{x}_j)),\bar{y}\}d\Psi_{j}(\mathbf{x}_j) = \sum_{j=J_R+1}^{J} \int_{\mathcal{X}} y_{ss} d\Psi_{j}(\mathbf{x}_j).
\end{align*}
}
\end{itemize}
\bigskip
\bigskip
\bigskip
\bigskip
\bigskip
\bigskip
\hyperlink{comp_eq}{\beamergotobutton{Competitive Equilibrium}}
\end{frame}




%%%%%%%%%%%%%%%%%%%%%%%%%%%%%%%%%%%%%%
\begin{frame}[label=calib]
\frametitle{Calibration}
\begin{table}[h!]
\footnotesize
\sf
\begin{center}
\begin{tabular}{p{1.5cm}| p{1.5cm} | p{7cm}}
\hline
Parameters  &	Value & Target \\ \hline\hline
\multicolumn{3}{l}{\textbf{Demographics}}\\\hline
$n$			&	0.011		& from BLS data\\
$J_R$       & 40 			& working age from 23 to 62\\
$J$	        & 58 			& age up to 80\\\hline
\multicolumn{3}{l}{\textbf{Preferences and Technology}}\\\hline
$r$ 		&0.042 			& annual net interest rate at 4.2\% \\
$\beta$		&0.966* 			& average capital-output ratio $K/Y=$2.947 \\
$\alpha$	& 0.322			& average capital income share\\
$\delta_k$	& 0.067 		& $\alpha \frac{Y}{K} - r$\\
$\gamma$	& [0.7,0.9]		& 0.7 as a benchmark from Huggett et al. 2011 \\
$\psi$		& 2.603*		& consumption age profile\\
$\phi$		& 4.387*		& labor supply age profile\\
$\chi$		& 0.052*		& average hours of work\\
$\delta_h$	& 0.023* 			& labor income decline in old age\\ \hline
\end{tabular}
\end{center}
\end{table}
\end{frame}


%%%%%%%%%%%%%%%%%%%%%%%%%%%%%%%%%%%%%%
\begin{frame}
\frametitle{Calibration}
\begin{table}[h!]
\footnotesize
\sf
\begin{center}
\begin{tabular}{p{1.5cm}| p{1.5cm} | p{7cm}}
\hline
Parameters  &	Value & Target \\ \hline\hline
\multicolumn{3}{l}{\textbf{Initial conditions}}\\\hline
$m_{h}$ 	& 2.734* 		& average initial wage at age 23 \\
$m_{\theta}$&-1.021*		& average wage growth \\
$\sigma_{h}$		&0.291*   & initial variance of wage profile \\
$\sigma_{\theta}$	&0.285* 	 & variance growth of wage profile \\
$\lambda_{h\theta}$ 		&0.003* & the covariance of initial wage and initial wage growth \\\hline
\multicolumn{3}{l}{\textbf{Shock process}}\\\hline
$\sigma_{\varepsilon}$ & 0.111 & Huggett et al. 2011\\\hline
\multicolumn{3}{l}{\textbf{Tax system }}\\\hline
$\tau_{ss}$ & 0.124 & Data\\
$\bar{y}$ & 2.5$\times E(y)$ & Data\\
%$y_{ss}$ & 1.002* & social security budget balance \\
$\tau_c$ & 0.06 & Mendoza et al. 1994\\
$\tau_k$ & 0.40 & Domeij \& Heathcote 2004\\
$\tau_{0}$ & 0.258 & Gouveia \& Strauss 1994 \\
$\tau_{1}$ & 0.768 & Gouveia \& Strauss 1994 \\
%$\tau_{2}$ & 0.000005* & government budget balance \\
\hline
\end{tabular}
\end{center}
\end{table}
\hyperlink{comp_eq}{\beamergotobutton{Competitive Equilibrium}}
\end{frame}




%%%%%%%%%%%%%%%%%%%%%%%%%%%%%%%%%%%%%%
\setcounter{subfigure}{0}
\begin{frame}[label=match]
\frametitle{Competitive Equilibrium Mean Profiles}
\begin{figure}
\begin{center}
\vspace{-0.5cm}
\subfloat[\sf Hourly Wage]{\includegraphics[width=.4\textwidth]{Graphs/mean_w_model2.png}\label{fig:mwage}}
\hspace{2cm}
\subfloat[\sf Consumption]{\includegraphics[width=.4\textwidth]{Graphs/mean_c_model2.png}\label{fig:mcons}}
\vspace{-0.2cm}
\subfloat[\sf Hours of Work]{\includegraphics[width=.4\textwidth]{Graphs/mean_n_model2.png}\label{fig:mlab}}
\end{center}
\end{figure}
\end{frame}




%%%%%%%%%%%%%%%%%%%%%%%%%%%%%%%%%%%%%%
\setcounter{subfigure}{0}
\begin{frame}
\frametitle{{\small Competitive Equilibrium Variance Profiles}}
\begin{figure}
\begin{center}
\vspace{-0.5cm}
\subfloat[\sf Hourly Wage]{\includegraphics[width=.4\textwidth]{Graphs/var_w_model2.png}\label{fig:vwage}}
\hspace{2cm}
\subfloat[\sf Consumption]{\includegraphics[width=.4\textwidth]{Graphs/var_c_model2.png}\label{fig:vcons}}
\vspace{-0.2cm}
\subfloat[\sf Hours of Work]{\includegraphics[width=.4\textwidth]{Graphs/var_n_model2.png}\label{fig:vlab}}
\end{center}
\end{figure}
\hyperlink{comp_eq}{\beamergotobutton{Competitive Equilibrium}}
\end{frame}



%%%%%%%%%%%%%%%%%%%%%%%%%%%%%%%%%%%%%%
\begin{frame}
\frametitle{Computational Algorithm: Household's Problem}
{\footnotesize
\begin{align*}
&\frac{u_c(c_j,1-n_j)}{1+\tau_c} = \lambda^c_j\\ 
&u_n(c_j,1-n_j) = \lambda^c_j w h_j (1-s_j) (1-0.5 \tau_{ss} \mathbb{I}_{\tilde{y}_j\leq \bar{y}}) (1-T'(\tilde{y}_j)) + \lambda^h_j \exp(\varepsilon_{j+1}) \frac{\partial H}{\partial n_j}\\
&\lambda^c_j wh_j n_j(1-0.5 \tau_{ss} \mathbb{I}_{\tilde{y}_j\leq \bar{y}}) (1-T'(\tilde{y}_j))  = \lambda^h_j \exp(\varepsilon_{j+1}) \frac{\partial H}{\partial s_j}\\
&(1+\tau_c) c_j + a_{j+1} = \tilde{y}_j - T(\tilde{y}_j)  + (1+r(1-\tau_{k}))a_j\\
&h_{j+1} = \exp(\varepsilon_{j+1})H(h_j,n_j,s_j;\theta_0)\\
&\beta \mathbb{E} V^a_{j+1} = \lambda^c_j \\
&\beta \mathbb{E} V^h_{j+1} = \lambda^h_j \\
&V^a_j = \lambda^c_j (1+r(1-\tau_{k})) \\
&V^h_j = \lambda^c_j w n_j(1-s_j) (1-0.5 \tau_{ss} \mathbb{I}_{\tilde{y}_j\leq \bar{y}}) (1-T'(\tilde{y}_j)) + \lambda^h_j \exp(\varepsilon_{j+1})\frac{\partial H}{\partial h_j}  \\
&\frac{\partial H}{\partial n_j} = \gamma \theta_0 (h_j n_j s_j)^{\gamma-1} h_j s_j,\quad
\frac{\partial H}{\partial s_j} = \gamma \theta_0 (h_j n_j s_j)^{\gamma-1}h_j n_j \\
&\frac{\partial H}{\partial h_j} = (1-\delta_h) + \gamma \theta_0 (h_j n_j s_j)^{\gamma-1} n_j s_j \\
\end{align*}
}
\end{frame}


%%%%%%%%%%%%%%%%%%%%%%%%%%%%%%%%%%%%%%
\begin{frame}
\frametitle{Computational Algorithm: Household's Problem}
{\footnotesize
\begin{enumerate}
\item[\underline{Step 1}.] For the endogenous grid methods, define grid points on $(a_{j+1},h_{j},\theta_0) \in \tilde{\mathcal{A}}\otimes\tilde{\mathcal{H}}\otimes\tilde{\Theta}_0$ for $j\leq J_R$ and on $a_{j+1} \in \tilde{\mathcal{A}}$ for $j > J_R$, where a set $\tilde{X}$ denotes a set of $N_x$ discretized points of a convex set $X$, $\tilde{X}=\{x_{min}=x_{0},x_{1},...,x_{N_x-1},x_{N_x}=x_{max}\}$. For the exogenous grid methods, define grid points on $(a_{j},h_{j},\theta_0) \in \tilde{\mathcal{A}}\otimes\tilde{\mathcal{H}}\otimes\tilde{\Theta}_0$ for $j\leq J_R$ and on $a_{j} \in \tilde{\mathcal{A}}$ for $j > J_R$.
\item[\underline{Step 2}.] [\textit{EXGM}] At $j=J$, $a_{J+1}=0$ for any given $a_{J}$. Therefore, for each grid point $a_{J}\in \tilde{\mathcal{A}}$, we obtain 
\begin{align*}
&c^*_J = (y_{ss} + (1+r(1-\tau_{k})) a_J) / (1+\tau_c)\\
&W^a_{J}(a_{J}) = u'(c^*_J)(1+r(1-\tau_{k}))/ (1+\tau_c)
\end{align*}
\end{enumerate}
}
\end{frame}



%%%%%%%%%%%%%%%%%%%%%%%%%%%%%%%%%%%%%%
\begin{frame}
\frametitle{Computational Algorithm: Household's Problem}
{\footnotesize
\begin{enumerate}
\item[\underline{Step 3}.] [\textit{ENGM}\&\textit{EXGM}] At $J_R+1\leq j<J$, we find $(\hat{c}_j, \hat{a}_j)$ for each grid point $a_{j+1}\in \tilde{\mathcal{A}}$:
\begin{align*}
u'(\hat{c}_j) &= (1+\tau_c)\beta W^a_{j+1}(a_{j+1})\\
\hat{a}_j &= ((1+\tau_c) \hat{c}_j + a_{j+1} - y_{ss})/R(1-\tau_{k}).
\end{align*}
and using a derived policy function $g_j(\hat{a}_j) = a_{j+1}$, interpolate a policy function $g_j(a_j) = a^*_{j+1}$ with respect to each exogenous grid point, $a_j\in \tilde{\mathcal{A}}$. Finally, if $a^*_{j+1}\geq -\underline{a}$, proceed to the $j-1$ problem. Otherwise, find $a^*_{j+1}$ and $c^*_j$ for the particular exogenous states $a_j\in\tilde{\mathcal{A}}\backslash  \tilde{\mathcal{A}}_{\{a_j:a^*_{j+1}\geq -\underline{a}\}}$ using a nonlinear solution method.
\item[\underline{Step 4}.] At $j=J_R$, it is optimal to invest no time in human capital, $s_j = 0$.\footnote{\sf From $j=J_R$ to $j=J_R+1$, there is a transformation of the state space from three dimensions to one dimension. That is, the household problem at $j=J_R$ becomes
\begin{align*}
V_j(a_j,h_j;\theta_0) &= \max_{c_j,a_{j+1},s_j} u(c_j,1-n_j) +  \beta W_{j+1}(a_{j+1}) \\
\text{s.t.} \quad &(1+\tau_c) c_j + a_{j+1} = \tilde{y}_j - T(\tilde{y}_j)  + (1+r(1-\tau_{k}))a_j \\
&\tilde{y}_j =w h_j n_j (1-s_j) - 0.5\tau_{ss}\min(w h_j n_j(1-s_j),\bar{y}) \\
&a_j \geq -\underline{a}.
\end{align*}
As the stock of human capital accumulated up until $J_R$ and initial productivity becomes irrelevant to the benefits after retirement, it is trivial that $s_{J_R} = 0$. Therefore, the problem is identical to that in Step 3 except for the envelope condition for human capital $V^h_j$.
} Otherwise, the computational algorithm follows as in step 5.
\end{enumerate}
}
\end{frame}



%%%%%%%%%%%%%%%%%%%%%%%%%%%%%%%%%%%%%%
\begin{frame}
\frametitle{Computational Algorithm: Household's Problem}
{\footnotesize
\begin{enumerate}
\item[\underline{Step 5}.] [\textit{ENGM}\&\textit{EXGM}] At $j<J_R$, for each grid point $(a_{j+1},h_{j},\theta_0) \in  \tilde{\mathcal{X}}$, we first numerically solve nonlinear equations for $\hat{n}_j$ and $\hat{s}_j$. An optimal pair of $(\hat{n}_j,\hat{s}_j)$ with $V^a_{j+1}$ and $V^h_{j+1}$ determines Lagrange multipliers $\hat{\lambda}^c_j$ and $\hat{\lambda}^h_j$.

Then, we find $\hat{c}_j$. With three policy functions $g^n_j(\hat{a}_j,h_j,\theta_0) = \hat{n}_j$, $g^s_j(\hat{a}_j,h_j,\theta_0) = \hat{s}_j$, and $g^a_j(\hat{a}_j,h_j,\theta_0) = a_{j+1}$, interpolate new policy functions $g^n_j(a_j,h_j,\theta_0) = n^*_j$, $g^s_j(a_j,h_j,\theta_0) = s^*_j$, and $g^a_j(a_j,h_j,\theta_0) = a^*_{j+1}$ for each exogenous grid point, $(a_j,h_{j},\theta_0) \in  \tilde{\mathcal{X}}$. Finally, if $a^*_{j+1}\geq -\underline{a}$, proceed to the $j-1$ problem. Otherwise, find $n^*_j$ and $s^*_j$ for the particular exogenous states $(a_j,h_j,\theta_0)\in\tilde{\mathcal{X}}\backslash  \tilde{\mathcal{X}}_{\{(a_j,h_j,\theta_0):a^*_{j+1}\geq -\underline{a}\}}$ using a nonlinear solution method. Given the solutions $(n^*_j,s^*_j,a^*_{j+1})$, we compute envelope conditions $V^a_{j}$ and $V^h_{j}$ for each $(a_j,h_j,\theta_0)\in\tilde{\mathcal{X}}$.
\end{enumerate}
}
\end{frame}



%%%%%%%%%%%%%%%%%%%%%%%%%%%%%%%%%%%%%%
\begin{frame}
\frametitle{Computational Algorithm: Constrained Efficiency}
{\tiny
\begin{align*}
&\frac{u_c(c_j,1-n_j)}{1+\tau_c} = \beta \mathbb{E} V^a_{j+1} = \lambda^c_j\\ 
&V^a_j = \lambda^c_j (1+F_K (1-\tau_{k})) + \Delta_k \\
&\Delta_k = \sum_{i=0}^{J} \mu_i \int_{\chi} \lambda^c_i F_{KK} K \left[ \frac{(1-\tau_{k})a_i}{K} - \frac{h_i n_i (1-s_i)}{L}(1-0.5 \tau_{ss} \mathbb{I}_{\tilde{y}_i\leq \bar{y}}) (1-T'(\tilde{y}_i))\right] d\Psi_i  \\
&\lambda^c_j F_L h_j n_j (1-T'(\tilde{y}_j))(1-0.5 \tau_{ss} \mathbb{I}_{y_j\leq \bar{y}}) + h_j n_j \Delta_{h} = \beta \mathbb{E} V^h_{j+1} \exp(\varepsilon_{j+1})\frac{\partial H_j}{\partial s_j} \\ 
&\Delta_{h} = \sum_{i=0}^{J_R} \mu_i \int_{\chi} \lambda^c_i F_{KL}K \left[  \frac{(1-\tau_k)a_i}{K} - \frac{h_i n_i(1-s_i)}{L} (1-0.5 \tau_{ss} \mathbb{I}_{\tilde{y}_i\leq \bar{y}})(1-T'(\tilde{y}_i)) \right] d\Psi_i \\
&V^h_j = \lambda^c_j F_L n_j (1-s_j) (1-T'(\tilde{y}_j))(1-0.5 \tau_{ss} \mathbb{I}_{\tilde{y}_j\leq \bar{y}}) + n_j (1-s_j)\Delta_{h} + \beta\mathbb{E}  V^h_{j+1} \exp(\varepsilon_{j+1})\frac{\partial H_j}{\partial h_j}  \\
&u_n(c_j,1-n_j) = \lambda^c_j F_L h_j (1-s_j) (1-0.5 \tau_{ss} \mathbb{I}_{\tilde{y}_j\leq \bar{y}}) (1-T'(\tilde{y}_j)) + \lambda^h_j \exp(\varepsilon_{j+1}) \frac{\partial H}{\partial n_j} \\
&\quad\quad\quad\quad\quad\quad\quad\quad + h_j (1-s_j) \Delta_h
\end{align*}
}
\end{frame}



%%%%%%%%%%%%%%%%%%%%%%%%%%%%%%%%%%%%%%
\begin{frame}
\frametitle{Computational Algorithm: Constrained Efficiency}
\begin{itemize}
\item[Step 1.] Guess an initial distribution $\hat{\Psi} = \{\hat{\Psi}_1,\hat{\Psi}_2,\ldots,\hat{\Psi}_{J}\}$. A good initial guess could be a competitive equilibrium distribution.
\item[Step 2.] Compute the value functions and policy functions as in the competitive equilibrium and simulate the distribution $\tilde{\Psi}$.
\item[Step 3.] If $|\hat{\Psi} - \tilde{\Psi}|<\delta_{\Psi}$, then stop. Otherwise, repeat the process with the new distribution $\tilde{\Psi}$.
\end{itemize}
\end{frame}







\end{document}


